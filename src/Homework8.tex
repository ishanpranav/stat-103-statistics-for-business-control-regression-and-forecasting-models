\documentclass[12pt]{article}
\usepackage[english]{babel}
\usepackage[letterpaper,top=2cm,bottom=2cm,left=3cm,right=3cm,marginparwidth=1.75cm]{geometry}
\usepackage{amsmath}
\usepackage{graphicx}
\title{STAT-UB 103 Homework 8}
\author{Ishan Pranav}
\date{April 23, 2023}
\renewcommand{\thesubsection}{\thesection.\alph{subsection}}
\renewcommand{\theenumi}{\alph{enumi}}
\begin{document}
\maketitle
\section{Learning the mechanics}
\begin{enumerate}
\item See below.
\begin{center}
\begin{tabular}{ccccccc}
$x_i$&&$y_i$&&$x^2_i$&&$x_iy_i$\\\\
7&&2&&49&&14\\
4&&4&&16&&16\\
6&&2&&36&&12\\
2&&5&&4&&10\\
1&&7&&1&&7\\
1&&6&&1&&6\\
3&&5&&9&&15\\\\
$\sum_{i=0}^6{x_i}=24.$&&$\sum_{i=0}^6{y_i}=31.$&&$\sum_{i=0}^6{x^2_i}=116.$&&$\sum_{i=0}^6{x_iy_i}=80.$\\
\end{tabular}
\end{center}
\item\[s^2_{x,y}=\sum^{n-1}_{i=0}{(x_i-\bar{x})(y_i-\bar{y})}\approx -26.2857\dots\]
\item\[s^2_x=\sum^{n-1}_{i=0}{(x_i-\bar{x})^2}\approx 33.7143\dots\]
\item\[b_1=\frac{s^2_{x,y}}{s^2_x}\approx -0.7797\dots\]
\item\[\bar{x}\approx 3.4286\dots\]
\[\bar{y}\approx 4.4286\dots\]
\item\[\bar{y}=b_1\bar{x}+b_0.\]

\[b_0=\bar{y}-b_1\bar{x}\approx 7.1017\dots\]
\item\[\hat{y}=b_1x+b_0=-0.7797x+7.1017.\]
\end{enumerate}
\section{Forecasting movie revenues with Twitter}
\[b_1=0.078767\dots\]
Assuming that movie revenue and tweet rate are linearly related, we estimate a movie's opening weekend revenue increases by an average of 7,876,700 dollars as the tweet rate for the movie increases by an average of 100 tweets per hour.
\section{Congress voting on women's issues}
Let $y$ represent a legislator's American Association of University Women score as a function of the number of daughters ($x$) that the legislator has.

$y=\beta_1x+\beta_0.$

\begin{enumerate}
\item If it is true that having a daughter influences voting on women's issues, the sign of $\beta_1$ will be positive. A positive linear coefficient ($\beta_1$) indicates a positive linear relationship.
\item
\[n=434.\]
\[\nu=n-2=432.\]
\[b_1\approx 0.27.\]
\[s_{b_1}\approx 0.74.\]
\[t^*=F^{-1}_{432}(95\%)=1.6484\dots\]

We can be ninety-five-percent confident that the true value of $\beta_1$ is in the interval $(-0.95,1.49)$.
\end{enumerate}
\section{RateMyProfessors.com}
Let $y$ represent the student evaluation of teaching and $x$ represent the RateMyProfessors.com rating.
\[n=426.\]
\[r_{x,y}\approx 0.68.\]
\begin{enumerate}
\item
\[r_{x,y}=\frac{s^2_{x,y}}{s_xs_y}=\frac{b_1s_x}{s_y}.\]
\[b_1=\frac{r_{x,y}s_y}{s_x}.\]

\[\hat{y}=\frac{0.68s_y}{s_x}+b_0.\]
\item The sample correlation of 0.68 indicates a strong positive linear relationship between the RateMyProfessors.com rating and the student evaluation of teaching. Students who rate their professors highly on RateMyProfessors.com also tend to give high evaluations of teaching.
\item The estimated slope of the line is positive. A positive sample correlation indicates a positive linear relationship.
\item There is convincing evidence, at the 0.1\% significance level, that RateMyProfessors.com ratings and student evaluations of teaching are correlated.
\item\[r^2\approx 0.46.\]

Approximately 46 percent of the variation in student evaluations of teaching can be explained by variations in RateMyProfessors.com ratings.
\end{enumerate}
\end{document}